\documentclass{article}
\usepackage[utf8]{inputenc}

\begin{document}

\section*{Prompt para explicación del Jacobiano}

Explica en LaTeX por qué $dx\,dy = |J|\,d\xi\,d\eta$ en cambios de variable, como si le estuvieras enseñando a un compañero de clase. Incluye:

\begin{itemize}
    \item Una analogía simple (como estirar un chicle o una malla)
    \item La definición breve del Jacobiano y por qué aparece su determinante
    \item Un ejemplo numérico fácil (con números pequeños)
    \item Una advertencia sobre el error más común (como olvidar el valor absoluto)
\end{itemize}

Usa máximo 2 páginas, ecuaciones sencillas y un lenguaje relajado, pero preciso. No te pongas muy técnico, que se entienda la idea central.

\end{document}
