\documentclass{article}
\usepackage[utf8]{inputenc}
\usepackage{amsmath}
\usepackage{graphicx}

\title{Explicación del Jacobiano para Cambios de Variable}

\begin{document}

\maketitle

\section{¿Por qué $dx\,dy = |J|\,d\xi\,d\eta$?}

Imagina que tienes un chicle cuadrado pegado en una mesa. Si lo estiras con tus dedos, el área que cubre cambia. Eso es exactamente lo que hace el Jacobiano: mide cómo se "estira" o "comprime" el área cuando cambiamos de variables.4

\begin{itemize}
\item \textbf{Fundamento físico}: El Jacobiano cuantifica cómo se distorsiona el espacio bajo transformaciones.
\item \textbf{Relación con integrales}: Es el factor que preserva el valor de la integral al cambiar variables.
\item \textbf{Caso general}: Para $n$ dimensiones, el Jacobiano es el determinante de la matriz de derivadas parciales de orden $n\times n$.
\end{itemize}
\end{minipage}
}

\section{El Jacobiano en 1 Minuto}

El Jacobiano es una matriz que contiene todas las derivadas parciales:

\[
J = \begin{pmatrix}
\frac{\partial x}{\partial \xi} & \frac{\partial x}{\partial \eta} \\
\frac{\partial y}{\partial \xi} & \frac{\partial y}{\partial \eta}
\end{pmatrix}
\]

\[
\text{Interpretación geométrica: } |J| = \left|\frac{\partial(x,y)}{\partial(\xi,\eta)}\right| = \text{Área del paralelogramo generado por}
\]
\[
\text{los vectores } \frac{\partial \vec{r}}{\partial \xi} \text{ y } \frac{\partial \vec{r}}{\partial \eta}
\]

Su determinante $|J|$ nos dice cuánto se escala el área:

\begin{itemize}
\item Si $|J| = 2$, el área se duplica (como estirar el chicle al doble)
\item Si $|J| = 0.5$, el área se reduce a la mitad (como aplastar el chicle)
\end{itemize}

\begin{minipage}{\dimexpr\textwidth-2\fboxsep}
\begin{itemize}
\item \textbf{Singularidades}: Si $|J| = 0$, la transformación colapsa dimensiones.
\item \textbf{No linealidades}: Para transformaciones no lineales, $J$ varía punto a punto.
\end{itemize}
\end{minipage}

\section{Ejemplo Sencillo}

Cambio de variables:
\begin{align*}
x &= 2\xi \\
y &= 3\eta
\end{align*}

El Jacobiano es:
\[
J = \begin{pmatrix}
2 & 0 \\
0 & 3
\end{pmatrix}, \quad |J| = 6
\]

Un cuadrado de $1 \times 1$ en $(\xi,\eta)$ se convierte en un rectángulo de $2 \times 3$ en $(x,y)$, con área 6 veces mayor.
\subsection{Ejemplo Avanzado}
\begin{minipage}{\dimexpr\textwidth-2\fboxsep}
\begin{align*}
x &= \xi\cos\eta \\
y &= \xi\sin\eta
\end{align*}
Jacobiano en coordenadas polares:
\[
J = \begin{pmatrix}
\cos\eta & -\xi\sin\eta \\
\sin\eta & \xi\cos\eta
\end{pmatrix}, \quad |J| = \xi
\]
Interpretación: El factor $\xi$ explica por qué anillos más alejados del origen (mayor $\xi$) tienen mayor área.
\end{minipage}


\section{¡Cuidado!}

El error más común es olvidar el valor absoluto:
\begin{itemize}
\item El área siempre es positiva
\item Si el determinante es negativo, usamos $|J|$ igual
\end{itemize}

\section{Resumen}

\begin{center}
\fbox{
\begin{minipage}{0.9\textwidth}
El Jacobiano corrige la distorsión del área cuando cambiamos de variables, como cuando estiramos un chicle. Solo recuerda usar su valor absoluto $|J|$.
\end{minipage}
}
\end{center}

\end{document}
